\documentclass[12pt,dvipdfmx]{article}
\setlength{\oddsidemargin}{-1.3truecm}
\setlength{\evensidemargin}{-1.3truecm}
\setlength{\textwidth}{18.5truecm}
\setlength{\headsep}{1truecm}
\setlength{\topmargin}{-2truecm}
\setlength{\textheight}{25truecm}
\usepackage{graphicx}
\DeclareGraphicsExtensions{.pdf}
\DeclareGraphicsExtensions{.eps}
\graphicspath{{out/}{out/tex/}{out/tex/gpl/}{out/tex/svg/}{out/tex/dot/}}
\usepackage{listings}
\usepackage{fancybox}
\usepackage{hyperref}
\usepackage{color}

\title{How much memory is needed}
\author{}
\date{}

\begin{document}
\maketitle

\section{Problem}
How much memory is needed to construct BWT of
$n$ characters using Hayashi's method?

\section{Leaf}
First we analyze leaf case.  
The flow of the algorithm for building BWT for range $[a,b)$ is:


\begin{center}
\begin{tabular}{|p{8cm}|l|l|}\hline
                                    & bits                  & lifetime \\\hline
build suffix array                  & $(b - a) \log n$      & 0 \\
build BWT based on the suffix array & $(b - a) \log \sigma$ & 2 \\
sample the suffix array             & $2S \log n$           & 1 \\
build character occurrence array    & $\sigma \log n$       & 1 \\
build wavelet matrix                & $(b - a) \log \sigma$ & 1 \\\hline
\end{tabular}
\end{center}

note:
\begin{itemize}
\item lifetime column indicates whether
  the bits are required:
  \begin{itemize}
  \item (0): only temporarily; can be freed at some point 
    during constructing the BWT of this range
  \item (1): to represent the result of this range; can be
    freed after it is merged into its parent range
  \item (2): to represent the final result; cannot be
    freed until the BWT of the whole range has been constructed
  \end{itemize}

\item $S$ can be chosen; 
  in practice, $S$ will be $\in O(n / \log n)$,
  so that this part will take $\in O(n)$;
  this is a part of our augmented BWT.
  In theory, we should choose $S \in O(n / \log n)$, 
  to make the space for the sampled array $\in O(n)$.
  In practice, we choose $S = n / 64 + 2$.
\end{itemize}

\section{Merge}
Let's say we are merging two ranges $[a_1, b_1)$ and $[a_2, b_2)$.
The overall structure of merge is this.

\begin{center}
\begin{tabular}{|p{8cm}|l|l|}\hline
 & bits & lifetime \\\hline
build gap array          & $(b_1 - a_1 + 1) (A + (1/B + 1/C) \log n)$ & 0 \\
sort right samples       & $S \log n$                         & 0 \\
scan the right BWT to fill the gap array & -                      & - \\
prefix sum the gap array & $((b_2 - a_1) / G) \log n$                      & 0 \\
build BWT based on the suffix array (workspace) & $(b_2 - a_1) \log \sigma$ & 0 \\
build BWT based on the suffix array (result) & $(b_2 - a_1) \log \sigma$ & 2 \\
resample sampled arrays  & $2S \log n$                             & 1 \\
build character occurrence array & $\sigma \log n$              & 1 \\
build wavelet matrix     & $(b_2 - a_1) \log \sigma$             & 1 \\
array of succinct bit vectors & & \\
succinct bit vectors     &  & \\
partition by bit         &  & \\
\end{tabular}
\end{center}

note:
\begin{itemize}
\item $A$, $B$, and $C$ can be chosen.
\item $A$ is typically 8, meaning a single byte is used to 
maintain a non-overflowing counts in a gap array
\item $1/B$ term is required to maintain
  overflowing counters in a gap array. it is
  chosen large enough to accommodate the
  worst-case number of overflows; that is,
  since the total counts put into the gap
  array is $(b_2 - a_2)$, and a counter
  overflows at $2^A - 2$\footnote{we reserve
    $(2^A - 1)$ to indicate an overflowed
    entry}, the overflow array must have at
  least $(b_2 - a_2) / (2^A - 1)$ entries.
  so in practice, assuming $A = 8$, we use $B = 128$.
  (they are currently hardcoded).

\item $1/C$ term is required to maintain a
  prefix sum of the gap array, to quickly
  find the place in left to insert each right
  element into.  We do not have a luxury of
  the full prefix array as it would need
  $\Omega (n \log n)$ memory.  In theory,
  we should choose $C \in \Omega (\log n)$.
  In practice, we choose $C = 128$
  ({\tt gap\_sum\_gran}).

\end{itemize}


\end{document}
